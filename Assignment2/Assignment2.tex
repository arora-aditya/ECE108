\documentclass[10.5pt]{article}

\usepackage{answers,setspace,graphicx,multicol,enumitem,textcomp}
\usepackage{mathrsfs,svg,adjustbox}
\usepackage[margin=0.5in]{geometry}
\usepackage{geometry}
\usepackage{totcount,xcolor,layout,latexsym,times,subfigure}
\usepackage{epsf,prooftrees,natded}
\input{epsf}
\usepackage[normalem]{ulem}
\usepackage{amsmath,amsthm,amssymb,xspace,multirow,graphicx,turnstile}
\usepackage[nocenter]{qtree}
\usepackage[titlenotnumbered,noend,noline]{algorithm2e}

\newcommand{\N}{\mathbb{N}}
\newcommand{\Z}{\mathbb{Z}}
\newcommand{\C}{\mathbb{C}}
\newcommand{\R}{\mathbb{R}}
\newcommand{\Q}{\mathbb{Q}}
\newcommand{\Jas}{Ja\'skowski }
\newcommand{\JasA}{Ja\'skowski}
\newcommand{\KM}{Kalish-Montague }
\newcommand{\KMa}{Kalish-Montague}

\DeclareMathOperator{\sech}{sech}
\DeclareMathOperator{\csch}{csch}

\newenvironment{theorem}[2][Theorem]{\begin{trivlist}
\item[\hskip \labelsep {\bfseries #1}\hskip \labelsep {\bfseries #2.}]}{\end{trivlist}}
\newenvironment{definition}[2][Definition]{\begin{trivlist}
\item[\hskip \labelsep {\bfseries #1}\hskip \labelsep {\bfseries #2.}]}{\end{trivlist}}
\newenvironment{proposition}[2][Proposition]{\begin{trivlist}
\item[\hskip \labelsep {\bfseries #1}\hskip \labelsep {\bfseries #2.}]}{\end{trivlist}}
\newenvironment{lemma}[2][Lemma]{\begin{trivlist}
\item[\hskip \labelsep {\bfseries #1}\hskip \labelsep {\bfseries #2.}]}{\end{trivlist}}
\newenvironment{exercise}[2][Exercise]{\begin{trivlist}
\item[\hskip \labelsep {\bfseries #1}\hskip \labelsep {\bfseries #2.}]}{\end{trivlist}}
\newenvironment{solution}[2][Solution]{ \begin{trivlist}
\item[\hskip \labelsep {\bfseries #1}]}{\end{trivlist}}
\newenvironment{problem}[2][Problem]{\begin{trivlist}
\item[\hskip \labelsep {\bfseries #1}\hskip \labelsep {\bfseries #2.}]}{\end{trivlist}}
\newenvironment{question}[2][Question]{\begin{trivlist}
\item[\hskip \labelsep {\bfseries #1}\hskip \labelsep {\bfseries #2.}]}{\end{trivlist}}
\newenvironment{corollary}[2][Corollary]{\begin{trivlist}
\item[\hskip \labelsep {\bfseries #1}\hskip \labelsep {\bfseries #2.}]}{\end{trivlist}}

\begin{document}

% --------------------------------------------------------------
%                         Start here
% --------------------------------------------------------------

\title{Assignment 2}%replace with the appropriate homework number
\author{Aditya Arora\\ %replace with your name
Winter 2018\\$Absque\ sudore\ et\ labore\ nullum\ opus\ perfectum\ est$} %if necessary, replace with your course title
\maketitle

%Below is an example of the problem environment
\begin{problem}{1}
Convert the following natural-language text into an equivalent set of propositional premises, together with an associated propositional conclusion.  Your answers should identify the text that corresponds to each proposition variable you use and the proposition formul{\ae} that correspond to the premises and the conclusion.  Determine if the argument is valid.
\begin{enumerate}
  \parskip=0in
  \parsep=0in
  \itemsep=0in
\item If the Big Bang Theory is correct then either there was a time before anything existed or the world will come to an end.  The world will not come to an end.  Therefore, if there was no time before anything existed, the Big Bang Theory is incorrect.
\item To win a gold medal, an athlete must be very fit.  If s/he does not win a gold medal, then either s/he arrived late for the competition or his/her training was interrupted.  If s/he is not very fit, s/he will blame his/her coach.  If s/he blames his/her coach, or his/her training is interrupted, then s/he will will not get into the competition.  Therefore, if s/he gets into the competition, s/he will not have arrived late.
\item If Hector wins the battle, he will plunder the city.  If he does not win the battle, he will either be killed or go into exile.  If he plunders the city, then Priam will lose his kingdom.  If Priam loses his kingdom or Hector goes into exile, then the war will end.  Therefore, if the war does not end, Hector will be killed.
\item If the colonel was out of the room when the murder was committed then he couldn't have been right about the weapon used.  Either the butler is lying or he knows who the murderer was.  If Lady Barntree was not the murderer then either the colonel was in the room at the time or or the butler is lying.  Either the butler knows who the murderer was or the colonel was out of the room at the time of the murder.  Therefore, if the colonel was right about the weapon then Lady Barntree was the murderer.
\end{enumerate}
\end{problem}
\begin{solution}{1}
\item[]
\begin{enumerate}
  \parskip=0in
  \parsep=0in
  \itemsep=0in
\item $P$: Big Bang Theory is correct, $Q$: There was a time before anything existed, $R$: The world will come to an end



\[
\frac{P\rightarrow \neg(Q \leftrightarrow R),\ \neg R,\ }{\neg Q \rightarrow \neg P}
\]
 $\KMproof{
  \cbblk{
  	\proofline{\neg Q \rightarrow \neg P}{}
  }{
    \proofline{\neg Q}{1 Ass CD}
    \cbblk{
      \proofline{\neg P}{}
    }{
        \proofline{P}{3 ID}
        \proofline{P\rightarrow \neg(Q \leftrightarrow R)}{P1}
        \proofline{\neg (Q \leftrightarrow R)}{4,5 MP}
        \proofline{\neg R}{P1}
        \proofline{Q}{5,6 MTP}
        \proofline{\neg Q}{2}
    }
  }
}$



% We know $\neg Q,\ \neg R$, and thus $(Q \oplus R)$ is $False$ from the first proposition. This clearly implies $\neg P$

\item $W$: Win a gold medal, $F$: Athlete is fit, $L$: Athlete arrived late for the competition, $I$: Training was interrupted, $B$: Athlete blames the coach, $C$: Athlete gets into the competition

\[
\frac{W \rightarrow F,\ \neg W \rightarrow \neg (L \leftrightarrow I),\ \neg (B \leftrightarrow I) \rightarrow \neg C}{C \rightarrow \neg L}
\]

The given conclusion is not valid. The counter argument for this is:
$W: T, F:T, L:T, I:F, B: F, C: T$ \\
The conclusions is false because $C \rightarrow \neg L$ is false but all the premises are true i.e.
\\ $W \rightarrow F: T$
\\ $\neg W \rightarrow (L \leftrightarrow I):T$
\\ $\neg F \rightarrow B: T$
\\ $\neg (B \leftrightarrow I) \rightarrow \neg C: T$
\\

\item $H$: Hector wins the battle, $P$: City is plundered by Hector, $K$: Hector gets killed, $E$: Hector goes into the exile, $L$: Priam will lose his kingdom, $W$: War will end  \\
\[
\frac{H \rightarrow P,\ \neg H \rightarrow (E \oplus K),\ P \rightarrow L,\ (L \oplus E) \rightarrow W}{\neg W \rightarrow K}
\]

 $\KMproof{
  \cbblk{
  	\proofline{\neg W \rightarrow K}{}
  }{
    \proofline{\neg W}{1 Ass CD}
    \cbblk{
      \proofline{K}{}
    }{
        \proofline{\neg (L \leftrightarrow E) \rightarrow W}{4th Premise}
        \proofline{(L \leftrightarrow E)}{2,4 Modus Tollens}
        \proofline{P \rightarrow L}{3rd Premise}
        \proofline{\neg L}{5 XOR}
        \proofline{\neg P}{3,5 Modus Tollens}
        \proofline{H \rightarrow P}{1st Premise}
        \proofline{\neg H}{8,9 Modus Tollens}
        \proofline{\neg E}{5 XOR}
        \proofline{\neg H \rightarrow \neg (E \leftrightarrow K)}{2nd Premise}
        \proofline{\neg (E \leftrightarrow K)}{10,12 Modus Ponens}
        \proofline{\neg(K \rightarrow E)}{13 BC}
        \proofline{K}{11,14 MP}
    }
  }
}$

Thus the given conclusion is true. This can also be established by backtracking the statements.\\


\item $O$: Colonel was out of the room when the murder was committed, $R$: Colonel is correct about the weapon used in the murder, $B$: Butler is lying, $K$: Butler knows who the murderer, $L$: Lady Barntree was the murderer
\[
\frac{O \rightarrow \neg R,\ \neg (B \leftrightarrow K),\ \neg L \rightarrow \neg(B \leftrightarrow \neg O),\ \neg (K \leftrightarrow O)}{R \rightarrow \neg L}
\]
The counter-model is: $O:F, R:F, B:F ,K:T, L:T$. All the premises are true but the conclusion is false

The given conclusion is false. This can be established by backtracking the statements.

We know $R$, this implies that $\neg O$ from the first proposition. Using $\neg O$ in the second last proposition we can say that $K$ is true. Using $K$ in the second proposition we can say that $B$ is false. Using all this we can say that $\neg(B \leftrightarrow \neg O)$ is $True$. But since $\neg L \rightarrow \neg(B \leftrightarrow \neg O)$ and $\neg(B \leftrightarrow \neg O)$ are both true we can not necessarily say that $\neg L$ is $True$
\end{enumerate}
\end{solution}

\vskip 0.5in
\newpage
\begin{problem}{2}
Which of the following are tautologies?  Which are contradictions?  Which are neither?  Justify your answer with truth tables, or sufficient models as is necessary.
\begin{enumerate}
  \parskip=0in
  \parsep=0in
  \itemsep=0in
\item $(P \wedge Q) \rightarrow (P \rightarrow Q)$
\item $(P \wedge Q) \leftrightarrow (P \rightarrow Q)$
\item $(\lnot P \vee Q) \rightarrow (P \rightarrow \lnot Q)$
\item $(((P \rightarrow Q) \rightarrow P) \rightarrow Q)$
\item $(P \rightarrow (Q \rightarrow (P \rightarrow Q)))$
\item $((P \wedge \lnot Q) \rightarrow \lnot R) \leftrightarrow ((P \wedge R) \rightarrow Q)$
\item $(((P \vee Q) \vee R) \vee S) \leftrightarrow (P \vee (Q \vee (R \vee S)))$
\item $(((P \rightarrow Q) \rightarrow R) \rightarrow S) \leftrightarrow (P \rightarrow (Q \rightarrow (R \rightarrow S)))$
\item $(P \rightarrow (\lnot R \rightarrow \lnot S)) \vee ((S \rightarrow (P \vee \lnot T)) \vee (\lnot Q \rightarrow R))$
\end{enumerate}
\end{problem}
\begin{solution}{2}
\item[]
\begin{table}[!h]
\centering
\caption{Part 1. Clearly a Tautology}
\label{my-label}
\begin{tabular}{|l|l|l|l|l|}
\hline
P & Q & $P \rightarrow Q$ & $P \wedge Q$ & $(P \wedge Q) \rightarrow (P \rightarrow Q)$ \\ \hline
F & F & T               & F          & T                                          \\ \hline
F & T & T               & F          & T                                          \\ \hline
T & F & F               & F          & T                                          \\ \hline
T & T & T               & T          & T                                          \\ \hline
\end{tabular}
\end{table}

\begin{table}[!h]
\centering
\caption{Part 2. Clearly a Satisfiable}
\label{my-label}
\begin{tabular}{|l|l|l|l|l|}
\hline
P & Q & $P \rightarrow Q$ & $P \wedge Q$ & $(P \wedge Q) \leftrightarrow (P \rightarrow Q)$ \\ \hline
F & F & T               & F          & F                                          \\ \hline
F & T & T               & F          & F                                          \\ \hline
T & F & F               & F          & T                                          \\ \hline
T & T & T               & T          & T                                          \\ \hline
\end{tabular}
\end{table}

\begin{table}[!h]
\centering
\caption{Part 3. Clearly satisfiable}
\label{my-label}
\begin{tabular}{|l|l|l|l|l|}
\hline
P & Q & $(\lnot P \vee Q)$ & $(P \rightarrow \lnot Q)$ & $(\lnot P \vee Q) \rightarrow (P \rightarrow \lnot Q)$ \\ \hline
F & F & T & T & T \\ \hline
F & T & T & T & T \\ \hline
T & F & F & T & T \\ \hline
T & T & F & F & F \\ \hline
\end{tabular}
\end{table}

\begin{table}[!h]
\centering
\caption{Part 4. Clearly satisfiable}
\label{my-label}
\begin{tabular}{|l|l|l|l|l|}
\hline
P & Q & $(P \rightarrow Q)$ & $((P \rightarrow Q) \rightarrow P)$ & $(((P \rightarrow Q) \rightarrow P) \rightarrow Q)$ \\ \hline
F & F & T & F & T \\ \hline
F & T & T & F & T \\ \hline
T & F & F & T & F \\ \hline
T & T & T & T & T \\ \hline
\end{tabular}
\end{table}

\begin{table}[!h]
\centering
\caption{Part 5. Clearly a Tautology}
\label{my-label}
\begin{tabular}{|l|l|l|l|l|}
\hline
P & Q & $(P \rightarrow Q)$ & $(Q \rightarrow (P \rightarrow Q))$ & $(P \rightarrow (Q \rightarrow (P \rightarrow Q)))$ \\ \hline
F & F & T & T & T \\ \hline
F & T & T & T & T \\ \hline
T & F & F & T & T \\ \hline
T & T & T & T & T \\ \hline
\end{tabular}
\end{table}

\begin{table}[!h]
\centering
\caption{Part 6. Clearly a Tautology}
\label{my-label}
\begin{tabular}{|l|l|l|l|l|l|}
\hline
P & Q & R & $((P \wedge \lnot Q) \rightarrow \lnot R)$ & $((P \wedge R) \rightarrow Q)$ & $((P \wedge \lnot Q) \rightarrow \lnot R) \leftrightarrow ((P \wedge R) \rightarrow Q)$ \\ \hline
T & T & T & T & T & T \\ \hline
T & T & F & T & T & T \\ \hline
T & F & T & F & F & T \\ \hline
T & F & F & T & T & T \\ \hline
F & T & T & T & T & T \\ \hline
F & T & F & T & T & T \\ \hline
F & F & T & T & T & T \\ \hline
F & F & F & T & T & T \\ \hline
\end{tabular}
\end{table}

\begin{table}[!h]
\centering
\caption{Part 7. Clearly a Tautology}
\label{my-label}
\begin{tabular}{|l|l|l|l|l|l|}
\hline
P & Q & R & $((P \vee Q) \vee R) \vee S)$ & $(P \vee (Q \vee (R \vee S))$ & $(((P \vee Q) \vee R) \vee S) \leftrightarrow (P \vee (Q \vee (R \vee S)))$ \\ \hline
T & T & T & T & T & T \\ \hline
T & T & F & F & F & T \\ \hline
T & F & T & F & F & T \\ \hline
T & F & F & F & F & T \\ \hline
F & T & T & F & F & T \\ \hline
F & T & F & F & F & T \\ \hline
F & F & T & F & F & T \\ \hline
F & F & F & F & F & T \\ \hline
\end{tabular}
\end{table}

\pagebreak

\begin{table}[!h]
\centering
\caption{Part 8. Clearly satisfiable}
\label{my-label}
\begin{tabular}{|l|l|l|l|l|l|l|}
\hline
P & Q & R & S & $((P \rightarrow Q) \rightarrow R) \rightarrow S)$ & $(P \rightarrow (Q \rightarrow (R \rightarrow S))$ & $(((P \rightarrow Q) \rightarrow R) \rightarrow S) \leftrightarrow (P \rightarrow (Q \rightarrow (R \rightarrow S)))$ \\ \hline
T & T & T & T & T & T & T \\ \hline
T & F & T & F & F & T & F \\ \hline
\end{tabular}
\end{table}

\begin{table}[!h]
\centering
\caption{Part 9. Clearly a Tautology}
\label{my-label}
\begin{tabular}{|l|l|l|l|l|l|l|l|}
\hline
P & Q & R & S & T & $(P \rightarrow (\lnot R \rightarrow \lnot S))$ & $(S \rightarrow (P \vee \lnot T))$ & $(P \rightarrow (\lnot R \rightarrow \lnot S)) \vee (S \rightarrow (P \vee \lnot T)) \vee (\lnot Q \rightarrow R))$ \\ \hline
T & T & T & T & T & T & T & T                                                                    \\ \hline
T & T & T & T & F & T & T & T                                                                    \\ \hline
T & T & T & F & T & T & T & T                                                                    \\ \hline
T & T & T & F & F & T & T & T                                                                    \\ \hline
T & T & F & T & T & F & T & T                                                                    \\ \hline
T & T & F & T & F & F & T & T                                                                    \\ \hline
T & T & F & F & T & T & T & T                                                                    \\ \hline
T & T & F & F & F & T & T & T                                                                    \\ \hline
T & F & T & T & T & T & T & T                                                                    \\ \hline
T & F & T & T & F & T & T & T                                                                    \\ \hline
T & F & T & F & T & T & T & T                                                                    \\ \hline
T & F & T & F & F & T & T & T                                                                    \\ \hline
T & F & F & T & T & T & T & T                                                                    \\ \hline
T & F & F & T & F & T & T & T                                                                    \\ \hline
T & F & F & F & T & T & T & T                                                                    \\ \hline
T & F & F & F & F & T & T & T                                                                    \\ \hline
F & T & T & T & T & T & T & T                                                                    \\ \hline
F & T & T & T & F & T & T & T                                                                    \\ \hline
F & T & T & F & T & T & T & T                                                                    \\ \hline
F & T & T & F & F & T & T & T                                                                    \\ \hline
F & T & F & T & T & T & T & T                                                                    \\ \hline
F & T & F & T & F & T & T & T                                                                    \\ \hline
F & T & F & F & T & T & T & T                                                                    \\ \hline
F & T & F & F & F & T & T & T                                                                    \\ \hline
F & F & T & T & T & T & F & T                                                                    \\ \hline
F & F & T & T & F & T & T & T                                                                    \\ \hline
F & F & T & F & T & T & T & T                                                                    \\ \hline
F & F & T & F & F & T & T & T                                                                    \\ \hline
F & F & F & T & T & T & F & T                                                                    \\ \hline
F & F & F & T & F & T & T & T                                                                    \\ \hline
F & F & F & F & T & T & T & T                                                                    \\ \hline
F & F & F & F & F & T & T & T                                                                    \\ \hline
\end{tabular}
\end{table}

\end{solution}


\pagebreak
\pagebreak
\newpage
\clearpage
\begin{problem}{3}
For the propositional formul{\ae} in Question~2, use Semantic Tableaux to show whether each is a tautology, contradiction, or neither.
\begin{enumerate}
  \parskip=0in
  \parsep=0in
  \itemsep=0in
\item $(P \wedge Q) \rightarrow (P \rightarrow Q)$
\item $(P \wedge Q) \leftrightarrow (P \rightarrow Q)$
\item $(\lnot P \vee Q) \rightarrow (P \rightarrow \lnot Q)$
\item $(((P \rightarrow Q) \rightarrow P) \rightarrow Q)$
\item $(P \rightarrow (Q \rightarrow (P \rightarrow Q)))$
\item $((P \wedge \lnot Q) \rightarrow \lnot R) \leftrightarrow ((P \wedge R) \rightarrow Q)$
\item $(((P \vee Q) \vee R) \vee S) \leftrightarrow (P \vee (Q \vee (R \vee S)))$
\item $(((P \rightarrow Q) \rightarrow R) \rightarrow S) \leftrightarrow (P \rightarrow (Q \rightarrow (R \rightarrow S)))$
\item $(P \rightarrow (\lnot R \rightarrow \lnot S)) \vee ((S \rightarrow (P \vee \lnot T)) \vee (\lnot Q \rightarrow R))$
\end{enumerate}
\end{problem}
\begin{solution}{3}
\item[]
\begin{enumerate}
\item {
\Tree
[.{$((P\land Q)\to(P\to Q)): F$}
    [.{$(P\land Q), \neg(P\to Q) $}
        [.{$P: T, Q: T $}
            [.{$P: T, \neg Q: T $\\x} ]
        ]
    ]
]\\
Therefore this statement is a Tautology
}
\item {
\Tree
[.{$(P \wedge Q) \leftrightarrow (P \rightarrow Q) : F$}
    [.{$(P \wedge Q): T, (P \rightarrow Q): F $}
        [.{$P:T, Q:T $}
            [.{$P:T, Q: F $\\x} ]
        ]
    ]
    [.{$(P \wedge Q): T, (P \rightarrow Q): F$}
        [.{$P:F$}
            [.{$Q:F$\\O} ]
        ]
        [.{$Q:T$}
            [.{$P:F, Q:F$\\x} ]
        ]
    ]
]\\
Therefore this statement is satisfiable
}
\item {
\Tree
[.{$(\lnot P \vee Q) \rightarrow (P \rightarrow \lnot Q): F $}
    [.{$(\lnot P \vee Q): T, (P \rightarrow \lnot Q): F$}
        [.{$P:F$} [.{$P:T$\\x} ] [.{$Q:T$\\O} ] ] [.{$Q:T$} [.{$P:T$\\x} ] [.{$Q:T$\\O} ] ]
    ]
]\\
Therefore this statement is satisfiable
}
\item {
\Tree
[.{$(((P \rightarrow Q) \rightarrow P) \rightarrow Q): F$}
    [.{$(P \rightarrow Q) \rightarrow P): T $}
        [.{$(P \rightarrow Q): F $}
            [.{$P:T$\\O} ]
            [.{$Q:F$\\O} ]
        ]
        [.{$P: T$\\O} ]
    ]
    [.{$Q: F$} ]
]
}
{
\Tree
[.{$(((P \rightarrow Q) \rightarrow P) \rightarrow Q): T$}
    [.{$(P \rightarrow Q) \rightarrow P): F $}
        [.{$(P \rightarrow Q): T $}
            [.{$P:F$\\O} ]
            [.{$Q:T$\\O} ]
        ]
        [.{$P: F$\\O} ]
    ]
    [.{$Q: T$} ]
]\\
Since the proposition can be both True and False it is Satisfiable
}
\item {
\Tree [.{$(P \rightarrow (Q \rightarrow (P \rightarrow Q))):F$} [.{$P:T,\ (Q \rightarrow (P \rightarrow Q)):F$} [.{$Q:T,\ (P \rightarrow Q):F$} [.{$P:T,\ Q:F$\\x} ] ] ] ]\\
Therefore this statement is a Tautology
}
\item {
\Tree
[.{$((P \wedge \lnot Q) \rightarrow \lnot R) \leftrightarrow ((P \wedge R) \rightarrow Q)$}
    [.{$((P \wedge \lnot Q) \rightarrow \lnot R):F,\  ((P \wedge R) \rightarrow Q):T$}      [.{$(P \wedge \lnot Q):T,\ \lnot R:F $}
            [.{$P:T,\ Q:F,\ R:T$}
                [.{$(P \wedge R):F$}
                    [.{$P:F$\\x} ]
                    [.{$Q:F$\\x} ]
                ]
            [.{$Q:T$\\x} ] ]
        ]
    ]
    [.{$((P \wedge \lnot Q) \rightarrow \lnot R):T,\  ((P \wedge R) \rightarrow Q):F$}
        [.{$P \wedge R: T,\ Q:F$}
            [.{$P:T,R:T$}
                [.{$\lnot R: T$}
                    [.{$R: F$\\x} ]
                ]
                [.{$(P \wedge \lnot Q):F$}
                    [.{$P:F$} ]
                    [.{$\lnot Q:F $\\} {$Q:T$\\x} ]
                ]
            ]
        ]
    ]
 ]\\
Therefore this statement is a Tautology
}
\item {
\Tree
[.{$(((P \vee Q) \vee R) \vee S) \leftrightarrow (P \vee (Q \vee (R \vee S))):F$}
    [.{$(((P \vee Q) \vee R) \vee S):T,\ (P \vee (Q \vee (R \vee S))): F$}
        [.{$\lnot P:T,\ \neg (Q \lor (R\lor S)):T
        $}
            [.{$\lnot Q:T,\ \lnot (R \lor S):T $}
                [ .{$\neg R: T, \neg S: T$}
                    [.{$((P\lor Q)\lor R)$}
                        [.{$(P\lor Q)$} {$P$\\x} {$Q$\\x} ]
                        [.{$R$\\x} ]
                    ]
                    [.{$S$\\x} ]
                ]
            ]
        ]
    ]
    [.{$\neg(((P\lor Q)\lor R)\lor S):T, (P\lor(Q\lor(R\lor S))): T $}
        [.{$((P\lor Q)\lor R): F, \neg S: T $}
            [.{$\neg(P\lor Q): T, \neg R: F  $}
                [.{$\neg P: T, \neg Q: T $}
                    [.{$P: T$\\x} ]
                    [.{$(Q\lor(R\lor S)):T$}
                        [.{$Q:T$\\x} ]
                        [.{$R \lor S : T $}
                            [.{$R:T$\\x} ]
                            [.{$S:T$\\x} ]
                        ]
                    ]
                ]
            ]
        ]
    ]
]\\
Therefore this statement is a Tautology
}
\item {
\Tree
[.{$(((P \rightarrow Q) \rightarrow R) \rightarrow S) \leftrightarrow (P \rightarrow (Q \rightarrow (R \rightarrow S))): F $}
    [.{$(((P \rightarrow Q) \rightarrow R) \rightarrow S): T, P \rightarrow (Q \rightarrow (R \rightarrow S))): F $}
        [.{$P:T, (Q \rightarrow (R \rightarrow S)): F $}
            [.{$Q:T, (R \rightarrow S): F$}
                [.{$R:T, S:F $}
                    [.{$S:T$\\x} ]
                    [.{$((P \rightarrow Q) \rightarrow R):F $}
                        [.{$(P \rightarrow Q), R:F$\\x} ]
                    ]
                ]
            ]
        ]
    ]
    [.{$(((P \rightarrow Q) \rightarrow R) \rightarrow S): F, P \rightarrow (Q \rightarrow (R \rightarrow S))): T$}
        [.{$S:F,((P \rightarrow Q) \rightarrow R):T  $}
            [.{$(P \rightarrow Q):T$}
                [.{$P:F$\\O} ]
            ]
        ]
    ]
]\\Thus the statement is Satisfiable
}
\item {
\Tree
[.{$((P\to(\neg R\to \neg S))\lor((S\to(P\lor \neg T))\lor(\neg Q\to R))): False$}
    [.{$\neg(P\to(\neg R\to \neg S)): True, \neg((S\to(P\lor \neg T))\lor(\neg Q\to R)): True $}
        [.{$P: True, \neg(\neg R\to\neg S): True  $}
            [.{$\neg(S\to(P\lor\neg T)): True, \neg(\neg Q\to R): True  $}
                [.{$S, \neg(P\lor\neg T): True $}
                    [.{$\neg P: True, \neg\neg T: True$\\x} ]
                ]
            ]
        ]
    ]
]\\
\\
Therefore this statement is a Tautology
}
\end{enumerate}
\end{solution}

\vskip 0.5in
\newpage
\pagebreak
\pagebreak\newpage

\begin{problem}{4} For the propositional formul{\ae} in Question~2, for any case where the formula was a tautology, prove that it is a tautology with a Kalish-Montegue derivations.
\begin{enumerate}
  \parskip=0in
  \parsep=0in
  \itemsep=0in
\item $(P \wedge Q) \rightarrow (P \rightarrow Q)$
\item $(P \wedge Q) \leftrightarrow (P \rightarrow Q)$
\item $(\lnot P \vee Q) \rightarrow (P \rightarrow \lnot Q)$
\item $(((P \rightarrow Q) \rightarrow P) \rightarrow Q)$
\item $(P \rightarrow (Q \rightarrow (P \rightarrow Q)))$
\item $((P \wedge \lnot Q) \rightarrow \lnot R) \leftrightarrow ((P \wedge R) \rightarrow Q)$
\item $(((P \vee Q) \vee R) \vee S) \leftrightarrow (P \vee (Q \vee (R \vee S)))$
\item $(((P \rightarrow Q) \rightarrow R) \rightarrow S) \leftrightarrow (P \rightarrow (Q \rightarrow (R \rightarrow S)))$
\item $(P \rightarrow (\lnot R \rightarrow \lnot S)) \vee ((S \rightarrow (P \vee \lnot T)) \vee (\lnot Q \rightarrow R))$
\end{enumerate}
\end{problem}
\begin{solution}{4}
\item[]


{\textbf{1.}} $\KMproof{
  \cbblk{
  	\proofline{(P \wedge Q) \rightarrow (P \rightarrow Q)}{2--7 Conditionalization}
  }{
    \proofline{(P \wedge Q)}{1 Ass CD}
    \cbblk{
      \proofline{(P\rightarrow Q)}{4--5 Conditionalization}
    }{
        \proofline{P}{2 SIMP}
        \proofline{Q}{2 SIMP}
    }
  }
}$

{\textbf{5.}}$\KMproof{
  \cbblk{
  	\proofline{(P \rightarrow (Q \rightarrow (P \rightarrow Q)))}{2--7 Conditionalization}
  }{
    \proofline{P}{1 Ass CD}
    \cbblk{
      \proofline{(Q \rightarrow (P \rightarrow Q))}{3--7 Sub Derivation}
    }{
        \proofline{Q}{3 Ass CD}
        \cbblk{
            \proofline{(P\rightarrow Q)}{7--8 Sub Derivation}
        }{
        \proofline{P}{5 CD}
        \proofline{Q}{4 Repeat}
        }
    }
  }
}$
\pagebreak\\
{\textbf{6.}} Citing De-Morgan's Law proof from question 5\\$\KMproof{
  \cbblk{
  	\proofline{((P \wedge \lnot Q) \rightarrow \lnot R) \leftrightarrow ((P \wedge R) \rightarrow Q)}{2--21 Conditionalization}
  }{
    \cbblk{
      \proofline{((P \wedge \lnot Q) \rightarrow \lnot R) \rightarrow ((P \wedge R) \rightarrow Q)\ \ \ \ \ \ \ \ \ \ \ \ \  }{1 Subderivation}
    }{
        \proofline{((P \wedge \lnot Q) \rightarrow \lnot R)}{2 Ass CD}
        \cbblk{
            \proofline{((P \wedge R) \rightarrow Q)}{7--8 Sub-Derivation}
        }{
            \proofline{P \wedge R}{4 Conditional Derivation}
            \proofline{P}{5 Simplification}
            \proofline{R}{5 Simplification}
            \proofline{\neg \neg R}{7 Double Negation}
            \proofline{\neg (P \wedge \lnot Q)}{3,7 Modus Tollens}
            \proofline{\neg P \vee Q}{9 de-Morgan's law}
            \proofline{Q}{6,10 Modus Tollendo Ponens}
        }
    }
    \cbblk{
      \proofline{((P \wedge R) \rightarrow Q)\rightarrow ((P \wedge \lnot Q) \rightarrow \lnot R) }{1 Subderivation}
    }{
        \proofline{((P \wedge R) \rightarrow Q)}{12 Ass CD}
        \cbblk{
            \proofline{((P \wedge \lnot Q) \rightarrow \lnot R)}{7--8 Sub-Derivation}
        }{
            \proofline{P \wedge \lnot Q}{12 Conditional Derivation}
            \proofline{P}{5 Simplification}
            \proofline{\lnot Q}{5 Simplification}
            \proofline{\lnot(P \wedge R)}{13,5 Modus Tollens}
            \proofline{\lnot P \vee \lnot R}{18 de-Morgan's law}
            \proofline{\lnot R}{16,19 Modus Tollendo Ponens}
        }
    }
     \proofline{((P \wedge \lnot Q) \rightarrow \lnot R) \leftrightarrow ((P \wedge R) \rightarrow Q)}{2,12 Conditional to Biconditional, DD}
  }
}$


% {6.} $((P \wedge \lnot Q) \rightarrow \lnot R) \leftrightarrow ((P \wedge R) \rightarrow Q)$

{\textbf{7.}}
$$\textbf{Property 1: } \frac{\neg(P \vee Q)}{\neg P}$$
{
  \Tree
    [.{$\neg (P \vee Q):T, \neg P: F$}
        [.{$P: T$}
            [.{$P \vee Q: F$}
                [.{$P:F, Q:F$\\x} ]
            ]
        ]
    ]

}
$\KMproof{
  \cbblk{
  	\proofline{(((P \vee Q) \vee R) \vee S) \leftrightarrow (P \vee (Q \vee (R \vee S)))}{2--13 Conditionalization}
  }{
    \cbblk{
      \proofline{(((P \vee Q) \vee R) \vee S) \rightarrow (P \vee (Q \vee (R \vee S)))\ \ \ \ \ \ \ \ \ \ \ \ \  }{1 Subderivation}
    }{
        \proofline{(((P \vee Q) \vee R) \vee S)}{2 Ass CD}
        \cbblk{
            \proofline{(P \vee (Q \vee (R \vee S)))}{Sub derivation}
        }{
            \proofline{\neg (P \vee (Q \vee (R \vee S)))}{4 ID}
            \proofline{\neg P}{5 Property1}
            \proofline{\neg (Q \vee (R \vee S))}{5 Property1}
            \proofline{\neg Q}{7 Property1}
            \proofline{\neg (R \vee S)}{5 Property1}
            \proofline{\neg R}{9 Property1}
            \proofline{\neg S}{9 Property1}
            \proofline{(P \vee Q) \vee R}{3, 11 MTP}
            \proofline{P \vee Q}{10,13 MTP}
            \proofline{P}{8,13 MTP}
        }
    }
    \cbblk{
      \proofline{(P \vee (Q \vee (R \vee S))) \rightarrow  (((P \vee Q) \vee R) \vee S)}{1 Subderivation}
    }{
        \proofline{(P \vee (Q \vee (R \vee S)))}{15 Ass CD}
        \cbblk{
            \proofline{(((P \vee Q) \vee R) \vee S)}{18--8 Sub-Derivation}
        }{
            \proofline{\neg (((P \vee Q) \vee R) \vee S)}{17 ID}
            \proofline{\neg S}{18 Property1}
            \proofline{\neg ((P \vee Q) \vee R)}{18 Property1}
            \proofline{\neg R}{20 Property1}
            \proofline{\neg (P \vee Q)}{20 Property1}
            \proofline{\neg P}{22 Property1}
            \proofline{\neg Q}{22 Property1}
            \proofline{Q \vee (R \vee S)}{23,24 MTP}
            \proofline{(R \vee S)}{24,25 MTP}
            \proofline{R}{19,26 MTP}
        }
    }
    \proofline{(((P \vee Q) \vee R) \vee S) \leftrightarrow (P \vee (Q \vee (R \vee S)))}{2,15 CB,DD}
  }
}$

\newpage
{\textbf{9.}}

$$\textbf{Property 1: } \frac{\neg(P \vee Q)}{\neg P}$$
{
\begin{center}
  \Tree
    [.{$\neg (P \vee Q):T, \neg P: F$}
        [.{$P: T$}
            [.{$P \vee Q: F$}
                [.{$P:F, Q:F$\\x} ]
            ]
        ]
    ]
\end{center}
}

\begin{table}[!h]
\centering
\caption{Proof for $P \rightarrow Q = \lnot P \vee Q$}
\label{my-label}
\begin{tabular}{|l|l|l|l|}
\hline
P & Q & $P \rightarrow Q$ & $\lnot P \vee Q$ \\ \hline
T & T & T                 & T                \\ \hline
T & F & F                 & F                \\ \hline
F & T & T                 & T                \\ \hline
T & F & T                 & T                \\ \hline
\end{tabular}
\end{table}

$\textbf{Property 2: }P \rightarrow Q = \lnot P \vee Q $ \\ \\

Referencing De-Morgan's law from Question 5

$\KMproof{
  \cbblk{
  	\proofline{\neg(P \rightarrow (\lnot R \rightarrow \lnot S)) \vee (S \rightarrow (P \vee \lnot T)) \vee (\lnot Q \rightarrow R))}{2--9 Conditionalization}
  }{
        \proofline{\neg(P \rightarrow (\lnot R \rightarrow \lnot S)) \vee (S \rightarrow (P \vee \lnot T)) \vee (Q \vee R))}{1 ID}
        \proofline{\neg(P \rightarrow (\neg R \rightarrow \neg S))}{2 Property1}
        \proofline{\neg(\neg P \vee (\neg R \rightarrow \neg S))}{2 Property2}
        \proofline{P}{2 Property1}
        \proofline{\neg((S \rightarrow (P \vee \lnot T)) \vee (Q \vee R))}{2 Property1}
        \proofline{\neg(S \rightarrow (P \vee \lnot T))}{6 Property1}
        \proofline{\neg(\neg S \vee (P \vee \lnot T))}{7 Property2}
        \proofline{\neg(P \vee \neg T)}{8 Property1}
        \proofline{\neg P}{10 Property1}
        \proofline{P}{5}
  }
}$
\end{solution}

\vskip 0.5in
\pagebreak
\begin{problem}{5}If a set of premises is inconsistent, then attempting to prove things with these \\premises is necessarily useless. For example, given the (clearly inconsistent) premises:
\begin{enumerate}
  \parskip=0in
  \parsep=0in
  \itemsep=0in
\item $P$
\item $\lnot P$
\end{enumerate}
The proof for {\em any} statement $Q$ is then:\\
\begin{tabular}{lll}
1. & \sout{Show} $Q$ & \\
2. & \multicolumn{2}{l}{\multirow{2}{*}{
\begin{tabular}{|ll|}
\hline
 $P$ & Premise 1, ID \\
 $\lnot P$ & Premise 2 \\
\hline
\end{tabular}
}}\\
3. & \multicolumn{2}{l}{}\\
\end{tabular}\\
One technique for determining if a set of premises is inconsistent is to determine if their conjunction is a contradiction ({\it i.e.}, if there are $N$ premises, identified as $P_i$ for $i = 1 \mbox{ to } N$, then the premises are inconsistent if $\forall_{T_r}^N E_P(P_1 \wedge P_2 \wedge \ldots \wedge P_N) \rightarrow F$).
\begin{enumerate}
  \parskip=0in
  \parsep=0in
  \itemsep=0in
\item Considering all possible models to show that a set of premises is inconsistent will take $O(2^N)$ time, where $N$ is the number of propositional variables.  Proving inconsistency instead might be preferable.  However, as noted above, proving things with a set of inconsistent premises is necessarily useless.  How (precisely) can you prove a set of premises, $P_i$ for $i = 1 \mbox{ to } N$, is inconsistent using (a) Semantic Tableaux (b) Kalish-Montegue derivations?
\item It is possible that a set of premises is collectively a tautology ({\it i.e.}, if there are $N$ premises, identified as $P_i$ for $i = 1 \mbox{ to } N$, then $\vDash (P_1 \wedge P_2 \wedge \ldots \wedge P_N)$).  Does this cause a problem for proving things?  Are there any other implications of having a set of premises whose conjection is a tautology?
\item Consider the following set of premises: ``Sales of houses fall off if interest rates rise. Auctioneers are not happy if sales of houses fall off. Interest rates are rising. Auctioneers are happy.''
  \begin{enumerate}
  \item Formalize these premises into a set of propositional formul{\ae}.
  \item Demonstrate that they are inconsistent using truth tables
  \item Prove that they are inconsistent using a Kalish-Montegue derivation
  \end{enumerate}
\end{enumerate}
\end{problem}
\begin{solution}{5}
\item[]
\begin{enumerate}
    \item If our premises form a contradiction, then we can say that
$$\neg(P_1 \wedge P_2 \ldots \wedge P_N ) → T$$
We first convert it into a tautology then proceed normally with our Semantic Tableau\\
We would then use a semantic tableau to show contradiction occur if we start with:
$$\neg(P_1 \wedge P_2 \ldots \wedge P_N ) : F$$
which equals
$$ == (P_1 \wedge P_2 \ldots \wedge P_N ) : T$$
We will then as in any other ST, split the terms inside the dis-junctions showing that we do have a tautology. \\ This would this show the contradiction

We would also be able to use KM derivation ans we will again start with $$Show\ \ \neg(P_1 \wedge P_2 \ldots \wedge P_N )$$
which will evidently turn to
$$(P_1 \wedge P_2 \ldots \wedge P_N)\ \ \ \ 1,ID$$
If this turns out to be True we would have proved that this is a tautology.
But since we took the negation of our premises, we would have then proven that our premises form a contradiction

    \item Having the dis-junction of premises as a tautology shouldn't cause a problem but just that the logical implication of that premise will also turn out to be a tautology. This just means that the logical implication is also basically part of the premises themselves and that there is no need of an explicit proof
    \item \begin{enumerate}[label=(\alph*)]
  \itemsep=0in
\item $S$: Sales of houses fall off, $I$: Interest rates rise, $A$: Auctioneers are happy \\
Propositional formul\ae: $I \rightarrow S,\ S \rightarrow \lnot A,\ I,\ A$ \\ (They are in the same order as the given statements)
\item According to the problem, the premises are inconsistent if $\forall_{T_r}^N E_P(P_1 \wedge P_2 \wedge \ldots \wedge P_N) \rightarrow F$) for $N$ premises identified as $P_i$ for $i = 1 \mbox{ to } N$ \\ Therefore using the same property in our premises:\\
Conjunction of Premises $= (I \rightarrow S) \wedge (S \rightarrow \lnot A) \wedge I \wedge A$ \\

\begin{table}[!h]
\centering
\caption{The conjunction is clearly always false}
\label{my-label}
\begin{tabular}{|l|l|l|l|l|l|}
\hline
I & S & A & $(I \rightarrow S)$ & $S \rightarrow \lnot A$ & $(I \rightarrow S) \wedge (S \rightarrow \lnot A) \wedge I \wedge A$ \\ \hline
T & T & T & T & F & F \\ \hline
T & T & F & T & T & F \\ \hline
T & F & T & F & T & F \\ \hline
T & F & F & F & T & F \\ \hline
F & T & T & T & F & F \\ \hline
F & T & F & T & T & F \\ \hline
F & F & T & T & T & F \\ \hline
F & F & F & T & T & F \\ \hline
\end{tabular}
\end{table}

\item {$\KMproof{
  \cbblk{
  	\proofline{\lnot((I \rightarrow S) \wedge (S \rightarrow \lnot A) \wedge I \wedge A)}{2--8 Conditionalization}
  }{
    \proofline{I}{2 SIMP}
    \proofline{(I \rightarrow S)}{2 SIMP}
    \proofline{(S \rightarrow \lnot A)}{2 SIMP}
    \proofline{A}{2 SIMP}
    \proofline{\lnot (\lnot A)}{5 Double Negation}
    \proofline{\lnot S}{3,6 Modus Tollens}
    \proofline{\lnot I}{2,7 Modus Tollens}
  }
}$\\
Thus $((I \rightarrow S) \wedge (S \rightarrow \lnot A) \wedge I \wedge A)$ is a contradiction
}

\end{enumerate}
\end{enumerate}
\end{solution}


\vskip 1.5in
\newpage
\begin{problem}{6}Convert the following propositions to Conjunctive Normal Form (CNF) with at most 3 literals per clause
\begin{enumerate}
  \parskip=0in
  \parsep=0in
  \itemsep=0in
\item $(P \wedge Q) \rightarrow (P \rightarrow Q)$
\item $(P \wedge Q) \leftrightarrow (P \rightarrow Q)$
\item $(\lnot P \vee Q) \rightarrow (P \rightarrow \lnot Q)$
\item $(((P \rightarrow Q) \rightarrow P) \rightarrow Q)$
\item $(P \rightarrow (Q \rightarrow (P \rightarrow Q)))$
\item $((P \wedge \lnot Q) \rightarrow \lnot R) \leftrightarrow ((P \wedge R) \rightarrow Q)$
\item $(((P \vee Q) \vee R) \vee S) \leftrightarrow (P \vee (Q \vee (R \vee S)))$
\item $(((P \rightarrow Q) \rightarrow R) \rightarrow S) \leftrightarrow (P \rightarrow (Q \rightarrow (R \rightarrow S)))$
\item $(P \rightarrow (\lnot R \rightarrow \lnot S)) \vee ((S \rightarrow (P \vee \lnot T)) \vee (\lnot Q \rightarrow R))$
\end{enumerate}
\end{problem}
\begin{solution}{6}
\item[]
Using De-Morgan's law in all the proofs. Citing Proof here from question 7 for clarity. Also using the property shown below:
\begin{table}[!h]
\centering
\caption{Proof for $P \rightarrow Q = \lnot P \vee Q$}
\label{my-label}
\begin{tabular}{|l|l|l|l|}
\hline
P & Q & $P \rightarrow Q$ & $\lnot P \vee Q$ \\ \hline
T & T & T                 & T                \\ \hline
T & F & F                 & F                \\ \hline
F & T & T                 & T                \\ \hline
T & F & T                 & T                \\ \hline
\end{tabular}
\end{table}

$P \rightarrow Q = \lnot P \vee Q $ (Property) \\ \\
\begin{enumerate}
  \parskip=0in
  \parsep=0in
  \itemsep=0in
\item True
\item
$(P \wedge Q) \leftrightarrow (P \rightarrow Q)$\\
$= (P \wedge Q)\leftrightarrow(\neg P \vee Q)$\\
$= ((P \wedge Q)\rightarrow(\neg P \vee Q))\wedge((\neg P \vee Q)\rightarrow(P \wedge Q))$\\
$= (\neg (P \wedge Q)\vee(\neg P \vee Q))\wedge(\neg(\neg P \vee Q)\vee (P \wedge Q))$\\
$= (\neg P \vee  \neg Q)\vee (\neg P \vee Q))\wedge((P \wedge \neg Q)\vee (P \wedge Q))$\\
$= (\neg P \vee \neg Q \vee \neg P \vee Q)\wedge((P \wedge \neg Q)\vee (P \wedge Q)) (P \wedge \neg Q)\vee(P \wedge Q)$\\
$= ((P \wedge \neg Q)\vee P)\wedge((P \wedge \neg Q)\vee Q)$\\
$= ((P \vee P)\wedge(\neg Q \vee P)\wedge((P \vee Q)\wedge(\neg Q\vee Q)$\\
$= P \wedge(\neg Q\vee P)\wedge((P \vee Q)$\\
$= P$

\item $(\neg P \vee Q)\rightarrow (P \rightarrow \neg Q)$\\
$= \neg (\neg P \vee Q)\vee (P \rightarrow \neg Q)$\\
$= \neg (\neg P \vee Q)\vee (\neg P \vee \neg Q)$\\
$= (P \wedge \neg Q)\vee (\neg P \vee \neg Q)$\\
$= (P \vee (\neg P \vee \neg Q))\wedge ((\neg Q)\vee (\neg P \vee \neg Q))$\\
$= \neg P \vee  \neg Q$\\
\item $(((P \rightarrow Q)\rightarrow P)\rightarrow Q) $\\
$= (((\neg P \vee Q)\rightarrow P)\rightarrow Q) $\\
$= ((\neg (\neg P \vee Q)\vee P)\rightarrow Q) $\\
$= (\neg (\neg (\neg P \vee Q)\vee P)\vee Q) $\\
$= (\neg ((P \wedge  \neg Q)\vee P)\vee Q) $\\
$= (\neg ((P \vee P)\wedge  (\neg Q\vee P))\vee Q) $\\
$= ((\neg (P \vee P)\vee \neg (\neg Q\vee P))\vee Q) $\\
$= ((\neg P \vee (Q\wedge  \neg P))\vee Q) $\\
$= (((\neg P \vee Q)\wedge  (\neg P \vee \neg P))\vee Q) $\\
$= (((\neg P \vee Q)\wedge  \neg P)\vee Q) $\\
$= (\neg P \vee Q)\vee Q)\wedge  (\neg P \vee Q) $\\
$= (\neg P \vee Q)\wedge  (\neg P \vee Q)$\\
$= \neg P \vee  Q $
\item True
\item True
\item True
\item $(\lnot P \vee Q \vee S) \wedge (P \vee \lnot R \vee S)$\\

\begin{table}[!h]
\centering
\caption{Part 8. Clearly satisfiable}
\label{my-label}
\begin{tabular}{|l|l|l|l|l|}
\hline
P & Q & R & S & $(((P \rightarrow Q) \rightarrow R) \rightarrow S) \leftrightarrow (P \rightarrow (Q \rightarrow (R \rightarrow S)))$ \\ \hline
T & T & T & T & T \\ \hline
T & T & T & F & T \\ \hline
T & T & F & T & T \\ \hline
T & T & F & F & T \\ \hline
T & F & T & T & T \\ \hline
T & F & T & F & F \\ \hline
T & F & F & T & T \\ \hline
T & F & F & F & F \\ \hline
F & T & T & T & T \\ \hline
F & T & T & F & F \\ \hline
F & T & F & T & T \\ \hline
F & T & F & F & T \\ \hline
F & F & T & T & T \\ \hline
F & F & T & F & F \\ \hline
F & F & F & T & T \\ \hline
F & F & F & F & T \\ \hline
\end{tabular}
\end{table}

Using the truth table to derive the product of sums format: \\
$(\lnot P \vee Q \vee \lnot R \vee S) \wedge (\lnot P \vee Q \vee R \vee S) \wedge (P \vee \lnot Q \vee \lnot R \vee S) \wedge (P \vee Q \vee \lnot R \vee S)$\\
$= (\lnot P \vee Q \vee S \vee \lnot R) \wedge (\lnot P \vee Q \vee S \vee R) \wedge (P \vee \lnot R \vee S \vee \lnot Q) \wedge (P \vee \lnot R \vee S \vee Q)$ \ \ \ \ \ \ \ \ \ \ \ \  [$\vee$ is commutative/associative]\\
$= ((\lnot P \vee Q \vee S) \vee \lnot R) \wedge ((\lnot P \vee Q \vee S) \vee R) \wedge ((P \vee \lnot R \vee S) \vee \lnot Q) \wedge ((P \vee \lnot R \vee S) \vee Q)$\\
$= ((\lnot P \vee Q \vee S) \vee \lnot R) \wedge ((\lnot P \vee Q \vee S) \vee R) \wedge ((P \vee \lnot R \vee S) \vee \lnot Q) \wedge ((P \vee \lnot R \vee S) \vee Q)$\\
$= (((\lnot P \vee Q \vee S) \vee \lnot R) \wedge ((\lnot P \vee Q \vee S) \vee R)) \wedge (((P \vee \lnot R \vee S) \vee \lnot Q) \wedge ((P \vee \lnot R \vee S) \vee Q))$\\
$= (((\lnot P \vee Q \vee S)) \wedge (\lnot R \vee R)) \wedge (((P \vee \lnot R \vee S)) \wedge (\lnot Q \vee Q))$\ \ \ \ \ \ \ \ \ \ \ \ \ \  [De-Morgan's Law]\\
$= (\lnot P \vee Q \vee S) \wedge (P \vee \lnot R \vee S)$\ \ \ \ \ \ \ \ \ \ \ \ \ \  [$(\lnot Q \vee Q) : T$]
\item True
\end{enumerate}
\end{solution}


\vskip 0.5in
\pagebreak
\begin{problem}{7}
Define:
\[
\bigvee_{i = 1}^N P_i = P_1 \vee P_2 \ldots \vee P_N
\]
and
\[
\bigwedge_{i = 1}^N P_i = P_1 \wedge P_2 \ldots \wedge P_N
\]

Show that
\[
\bigvee_{i = 1}^N P_i \leftrightarrow \lnot \bigwedge_{i = 1}^N (\lnot P_i)
\]
and
\[
\bigwedge_{i = 1}^N P_i \leftrightarrow \lnot \bigvee_{i = 1}^N (\lnot P_i)
\]
are true for $N \in \mathbb{N}$

Hint: in addition to basic Kalish-Montegue derivation, you will need to add induction.
\end{problem}
\begin{solution}{7}
\item[]
\begin{table}[!h]
\centering
\caption{Proof using Truth Table of base case $P_{1} \wedge P_{2} = \lnot(\lnot P_{1} \vee \lnot P_{2})$}
\label{my-label}
\begin{tabular}{|l|l|l|l|l|}
\hline
$P_{1}$ & $P_{2}$ & $P_{1} \wedge P_{2}$ & $(\lnot P_{1} \vee \lnot P_{2})$ & $\lnot(\lnot P_{1} \vee \lnot P_{2})$ \\ \hline
T & T & T     & F                 & T                    \\ \hline
T & F & F     & T                 & F                    \\ \hline
F & T & F     & T                 & F                    \\ \hline
F & F & F     & T                 & F                    \\ \hline
\end{tabular}
\end{table}

Now let us assume $$P(y) =
\bigwedge_{i = 1}^y P_i = P_1 \wedge P_2 \ldots \wedge P_y$$\\
Assuming $$P(n) = \lnot \bigvee_{i = 1}^n (\lnot P_i)$$ as True we will try to show that $P(n+1)$ is True\\
We know $P(n) \wedge P_{n+1} = P(n+1)$
$$\Rightarrow P(n+1) = (\bigwedge_{i = 1}^n P_i) \wedge P_{n+1}  = (\lnot \bigvee_{i = 1}^n (\lnot P_i)) \wedge P_{n+1}$$
$$\Rightarrow P(n+1) = \lnot (\lnot (\lnot \bigvee_{i = 1}^n (\lnot P_i)) \vee (\lnot P_{n+1}))\ \  [Using\ the\ property\ proved\ from\ base\ case]$$
$$\Rightarrow P(n+1) = \lnot (\bigvee_{i = 1}^n (\lnot P_i) \vee (\lnot P_{n+1}))$$
$$\Rightarrow P(n+1) = \lnot (\bigvee_{i = 1}^{n+1} (\lnot P_i))$$
$$\Rightarrow P(n+1) = \lnot \bigvee_{i = 1}^{n+1} (\lnot P_i)$$

Thus we have proved $P(n+1)$ from $P(n)$, and thus using the Principle of Mathematical Induction we have proved \[
\bigwedge_{i = 1}^N P_i \leftrightarrow \lnot \bigvee_{i = 1}^N (\lnot P_i)
\]

\pagebreak

\begin{table}[!h]
\centering
\caption{Proof using Truth Table of base case $P_{1} \vee P_{2} = \lnot(\lnot P_{1} \wedge \lnot P_{2})$ }
\label{my-label}
\begin{tabular}{|l|l|l|l|l|}
\hline
$P_{1}$ & $P_{2}$ & $P_{1} \vee P_{2}$ & $(\lnot P_{1} \wedge \lnot P_{2})$ & $\lnot(\lnot P_{1} \wedge \lnot P_{2})$ \\ \hline
T & T & T     & F                 & T                    \\ \hline
T & F & T     & F                 & T                   \\ \hline
F & T & T     & F                 & T                    \\ \hline
F & F & F     & T                 & F                    \\ \hline
\end{tabular}
\end{table}

Now let us assume $$P(y) =
\bigvee_{i = 1}^y P_i = P_1 \vee P_2 \ldots \vee P_y$$\\
Assuming $$P(n) = \lnot \bigwedge_{i = 1}^n (\lnot P_i)$$ as True we will try to show that $P(n+1)$ is True\\
We know $P(n) \vee P_{n+1} = P(n+1)$
$$\Rightarrow P(n+1) = (\bigvee_{i = 1}^n P_i) \vee P_{n+1}  = (\lnot \bigwedge_{i = 1}^n (\lnot P_i)) \vee P_{n+1}$$
$$\Rightarrow P(n+1) = \lnot (\lnot (\lnot \bigwedge_{i = 1}^n (\lnot P_i)) \wedge (\lnot P_{n+1}))\ \  [Using\ the\ property\ proved\ from\ base\ case]$$
$$\Rightarrow P(n+1) = \lnot (\bigwedge_{i = 1}^n (\lnot P_i) \wedge (\lnot P_{n+1}))$$
$$\Rightarrow P(n+1) = \lnot (\bigwedge_{i = 1}^{n+1} (\lnot P_i))$$
$$\Rightarrow P(n+1) = \lnot \bigwedge_{i = 1}^{n+1} (\lnot P_i)$$

Thus we have proved $P(n+1)$ from $P(n)$, and thus using the Principle of Mathematical Induction we have proved \[
\bigvee_{i = 1}^N P_i \leftrightarrow \lnot \bigwedge_{i = 1}^N (\lnot P_i)
\]


\end{solution}
\vskip 0.5in

% \vskip 0.5in

% \begin{problem}{12}
% \item[]
% \begin{enumerate}[label=\alph*)]
% \end{enumerate}
% \end{problem}
% \begin{solution}{7}
% \item[]
% \begin{enumerate}[label=\alph*)]
% \end{enumerate}
% \end{solution}




\end{document}
