\documentclass[12pt]{article}

\usepackage{answers}
\usepackage{setspace}
\usepackage{graphicx}
\usepackage{enumitem}
\usepackage{multicol}
\usepackage{mathrsfs}
\usepackage[margin=1in]{geometry} 
\usepackage{amsmath,amsthm,amssymb}
 
\newcommand{\N}{\mathbb{N}}
\newcommand{\Z}{\mathbb{Z}}
\newcommand{\C}{\mathbb{C}}
\newcommand{\R}{\mathbb{R}}
\newcommand{\Q}{\mathbb{Q}}

\DeclareMathOperator{\sech}{sech}
\DeclareMathOperator{\csch}{csch}
 
\newenvironment{theorem}[2][Theorem]{\begin{trivlist}
\item[\hskip \labelsep {\bfseries #1}\hskip \labelsep {\bfseries #2.}]}{\end{trivlist}}
\newenvironment{definition}[2][Definition]{\begin{trivlist}
\item[\hskip \labelsep {\bfseries #1}\hskip \labelsep {\bfseries #2.}]}{\end{trivlist}}
\newenvironment{proposition}[2][Proposition]{\begin{trivlist}
\item[\hskip \labelsep {\bfseries #1}\hskip \labelsep {\bfseries #2.}]}{\end{trivlist}}
\newenvironment{lemma}[2][Lemma]{\begin{trivlist}
\item[\hskip \labelsep {\bfseries #1}\hskip \labelsep {\bfseries #2.}]}{\end{trivlist}}
\newenvironment{exercise}[2][Exercise]{\begin{trivlist}
\item[\hskip \labelsep {\bfseries #1}\hskip \labelsep {\bfseries #2.}]}{\end{trivlist}}
\newenvironment{solution}[2][Solution]{ \begin{trivlist}
\item[\hskip \labelsep {\bfseries #1}]}{\end{trivlist}}
\newenvironment{problem}[2][Problem]{\begin{trivlist}
\item[\hskip \labelsep {\bfseries #1}\hskip \labelsep {\bfseries #2.}]}{\end{trivlist}}
\newenvironment{question}[2][Question]{\begin{trivlist}
\item[\hskip \labelsep {\bfseries #1}\hskip \labelsep {\bfseries #2.}]}{\end{trivlist}}
\newenvironment{corollary}[2][Corollary]{\begin{trivlist}
\item[\hskip \labelsep {\bfseries #1}\hskip \labelsep {\bfseries #2.}]}{\end{trivlist}}
 
\begin{document}
 
% --------------------------------------------------------------
%                         Start here
% --------------------------------------------------------------
 
\title{Assignment 1}%replace with the appropriate homework number
\author{Aditya Arora\\ %replace with your name
Winter 2018} %if necessary, replace with your course title
 
\maketitle
%Below is an example of the problem environment
\begin{problem}{1}
Prove or Disprove:
\begin{enumerate}[label=\alph*)]
    \item $ R \subseteq \ S\ \Leftrightarrow \ R \ \subseteq ((S-T)\cup(R\ \cap\ T)) $
    \item $(A \cap B) \subseteq (B \cap C) \longrightarrow (A \subseteq C)$ 
    \item $A \in B \wedge B \in C \longrightarrow A \in C$
    \item $A \in B \wedge B \in C \longrightarrow A \subseteq C$
    \item $A \in B \wedge B \subseteq C \longrightarrow A \in C$
    \item $A \in B \wedge B \subseteq C \longrightarrow A \subseteq C$
\end{enumerate}
\end{problem}

%Below is the solution environment
\begin{solution}{1}
\item[] 
\begin{enumerate}[label=\alph*)]
\item This can be disproved by a counter example $$R = \{1,2,3,4\},\ S = \{1,2,3\},\ T = \{4\} \\$$
The right hand side is:
$R \subseteq \{1,2,3\} \cup \{4\}$ which is $R \subseteq \{1,2,3,4\} \\$ which is true, but on the other hand the left hand side $R \subseteq S$ is false and $R \nsubseteq S$ \\
Thus, the given statement is false

\item This can be disproved by a counter example $$A = \{1,2,3,4\},\ C = \{1,2,3\},\ B = \{1,2,3\} \\$$
The left hand side is:
$\{1,2,3\} \subseteq \{1,2,3\}$ which is true, but on the other hand the right hand side $A \subseteq B$ is false and $A \nsubseteq B$ \\
Thus, the given statement is false

\item This can be disproved by a counter example $$A = \{1,2,3,4\},\ B = \{\{1,2,3,4\},\{0\}\},\ C = \{\{\{1,2,3,4\},\{0\}\}\} \\$$
The left hand side is:
$A \in B \wedge B \in C$ which is true, but on the other hand the right hand side $A \in C$ is false and $A \notin C$ \\
Thus, the given statement is false

\item This can be disproved by a counter example $$A = \{1,2,3,4\},\ B = \{\{1,2,3,4\},\{0\}\},\ C = \{\{\{1,2,3,4\},\{0\}\}\} \\$$
The left hand side is:
$A \in B \wedge B \in C$ which is true, but on the other hand the right hand side $A \nsubseteq C$ is false and $A \nsubseteq C$ \\
Thus, the given statement is false

\item if $A \in B$ and $B \subseteq C$ then we can say that for every element x in $B$, x also exists in $C$ and since $A$ is an element of $B$ we can safely say that $A$ is an element of $C$ and hence $A \in C$

\item Repeating the proof in the previous part,\\ if $A \in B$ and $B \subseteq C$ then we can say that for every element x in $B$, x also exists in $C$ and since $A$ is an element of $B$ we can safely say that $A$ is an element of $C$ and hence $A \in C$ and therefore $A \subseteq C$
\end{enumerate}
\end{solution}

\vskip 0.5in

\begin{problem}{2}
Given sets $A$ and $B$, under what conditions does $A\ -\ B\ = B\ -\ A$?. Prove that this is the only conditions under which this is true
\end{problem}
\begin{solution}{2}
\item[] 
If $A - B$ = $B - A$ then \\ for any $x\in\ A - B$ = $B - A$ we know, $x \in A$ and $x \in B$ and $x\notin A$ and $x\notin B$.\\ That's a contradiction so $A - B$ = $B - A$ is empty.
\\  
Thus there are no elements in $A$ that are not in $B$. In other words $A$ is a subset of $B$. \\ Likewise there are no elements of $B$ that are in $A$. So $B$ is a subset of $A$.
\\
\indent So $A=B$
\end{solution}

\pagebreak


\begin{problem}{3}
\item[]
\begin{enumerate}[label=\alph*)]
    \item Given $ f \colon \ A \longrightarrow B$, what is the relation between $f$'s co-domain and it's image if $f$ is 
    \begin{enumerate}[label=(\roman*)]
    \item injective
    \item surjective
    \item bijective
    \end{enumerate}
    \item Given $ f \colon \ A \longrightarrow B$, what is the relation between the image of $f^{-1}$ and the domain of $f$ if $f$ is 
    \begin{enumerate}[label=(\roman*)]
    \item injective
    \item surjective
    \item bijective
    \end{enumerate}
\end{enumerate}
\end{problem}
\begin{solution}{3}
\item[]
\begin{enumerate}[label=\alph*)]
\item
    \begin{enumerate}[label=(\roman*)]
        \item injective : The image is a subset of the co-domain
        \item surjective: The image is equal to the co-domain
        \item bijective: The image is equal to the co-domain
        \end{enumerate}
\item
    \begin{enumerate}[label=(\roman*)]
        \item injective : The image is equal to the domain
        \item surjective: The inverse of $f$ is not defined
        \item bijective: The image is equal to the domain
        \end{enumerate}
\end{enumerate}
\end{solution}
\vskip 0.5in
\begin{problem}{4} Given sets $X,\ Y$ such that $X \subseteq Y$. Prove or disprove:
% \item[]
\begin{enumerate}[label=\alph*)]
    \item There exists an injection $ f \colon \ X \longrightarrow Y$
    \item There exists an surjection $ f \colon \ Y \longrightarrow X$
\end{enumerate}
\end{problem}
\begin{solution}{4}
\item[]
We are given that $X \subseteq Y$ thus, we can say cardinality($X$) $\leq$ cardinality($Y$)
\begin{enumerate}[label=\alph*)]
    \item From above we can clearly say $\forall x \in X\ \exists\ y \in Y$ such that such that each $x$ is mapped to a unique $y$ and there might be elements in Y that have no pre-image in X.\\
    Thus an injective function exists
    \item From above we can clearly say $\forall x \in X\ \exists\ y \in Y$ such that such that for each $x$ there exists a pre-image in Y.\\
    Thus a surjective function exists
\end{enumerate}
\end{solution}

\vskip 0.5in

\begin{problem}{5}Given the following functions, are they: injective, surjective, bijective, or none of these?
\item[]
\begin{enumerate}[label=\alph*)]
    \item $ f \colon \ \N \longrightarrow \N$ where
    $ f \colon \ x \mapsto x$
    \item $ g \colon \ \N \longrightarrow \N$ where
    $ g \colon \ x \mapsto x^2$
    \item $h \colon \ \Q^+ \longrightarrow \Q^+$ where
    $ f \colon \ x \mapsto 1/x$
    \item If it is possible to compose $f$ and $g$, $f$ and $h$ and $g$ and $h$ what is the result of the composition. If not explain why not?
\end{enumerate}
\end{problem}
\begin{solution}{5}
\item[]
\begin{enumerate}[label=\alph*)]
    \item f is bijective
    \item g is injective
    \item h is bijective
    \item 
    \begin{enumerate}[label=(\roman*)]
        \item $f \circ g \colon \N \longrightarrow \N$ where $ f \circ g \colon \ x \longrightarrow x^2$
        \item $f \circ h \colon \N \longrightarrow \Q^+$ where $ f \circ h \colon \ x \longrightarrow 1/x$
        \item $g \circ h \colon \N \longrightarrow \Q^+$ where $ g \circ h \colon \ x \longrightarrow 1/x^2$
    \end{enumerate}
\end{enumerate}
\end{solution}


\vskip 0.5in

\begin{problem}{6}Prove or disprove: Given a patial order ($X,<$), the pair ($X,<^{ref}$) is a partial order

\end{problem}
\begin{solution}{6}
\item[]
\begin{enumerate}[label=\alph*)]
    \item f is bijective
    \item g is injective
    \item h is bijective
    \item 
    \begin{enumerate}[label=(\roman*)]
        \item $f \circ g \colon \N \longrightarrow \N$ where $ f \circ g \colon \ x \longrightarrow x^2$
        \item $f \circ h \colon \N \longrightarrow \Q^+$ where $ f \circ h \colon \ x \longrightarrow 1/x$
        \item $g \circ h \colon \N \longrightarrow \Q^+$ where $ g \circ h \colon \ x \longrightarrow 1/x^2$
    \end{enumerate}
\end{enumerate}
\end{solution}

\pagebreak

\begin{problem}{7}
Consider the set of letter grades G = \{ A, B, C, D, F, INC, WD, AEG \}. A, B, C, D, and F are what you would intuitively think they are, with A “better than” B, which is “better than” C, etc. \\ \vskip 0.05in
INC means “Incomplete”. This means the final grade has not yet been assigned because some outstanding work is missing. After the outstanding work is done, a final grade will be assigned. If the outstanding work is never done, then an INC changes to an F after two terms. Is an INC “better than” an F? Well, it is at least “better than or equal to” an F, since it will never get worse than F, and let’s be honest: it’s really “better than” F because (a) it is easier to explain away an INC during an interview than an F (“I was sick an so did not complete ...; my averge prior to that was ¡state something plausible and consistent with other grades¿. ....”) (b) the student will probably do better than an F when the outstanding work is submitted (otherwise why bother do the work, since it will automatically change to an F in two terms); in the meantime, reason (a) says an F now is “worse than” an F later. Is INC “better than” a D? That is harder to argue. It could become an F. It means there is outstanding work and it will become an F if that work is not completed. We will therefore claim that INC is “better than or equal to” a D but cannot say more than that with respect to D. How about a B or C? Realistically, we cannot claim any relationship between INC and B or C, since we do not know what the result of completing the outstanding work will be, so there can be no “better than” relation between those grades and INC. Finally, there is A? Is A “better than” INC? Clearly it is, because there is no grade better than A. Even when the outstanding work is done, the best the INC can become is an A, and an A now means there is nothing to even explain in an interview. Ergo, A is “better than” INC. \vskip 0.2in
WD means “Withdrawn” and so the class was attended long enough that it will not be removed from a transcript, but not so long that a final-grade assessment will be made. It is somewhat awkward to have on a transcript (better to have dropped prior to the drop deadline), but far better than a failing grade on the transcript. Therefore WD is “better than” F. Is it better than D? Almost certainly it is. D is not a great grade. It is a pass, but barely. In a traditional setting, it means: pass, but this course cannot be used as a prerequisite for some subsequent course. If it is for an elective, unrelated to a major, it may be OK, but WD is arguably better since it says that the student recognized that it was better to drop the course than not. Ergo, WD is “better than” D. As with INC, it is not really comparable to C, since C is an OK, but not good, grade. How about B? B is a good grade. Is it better to have withdrawn from a course of got a B? This is unclear. It is probably reasonable to claim that B is “better than or equal to” a WD. \vskip 0.2in
Finally, AEG is a curious grade that hopefully you will never experience, but is useful in some circumstances. AEG is short for “Aegrotat” which is Latin for “s/he is ill.” It is a grade that is awarded when a student has demonstrated enough ability to show that s/he has passed the course, but no better granularity is possible on the grade because there is insufficient information to make that determination. It is almost like a “pass” in a “pass/fail” system, except that what is it really saying is “the grade is somewhere between A and D but we do not have enough information to say where (because the student was ill).” Given that it is often viewed as a “pass” in a “pass/fail” system, it is reasonable to argure that an AEG is “at least as good as” a C, but not much more can be stated about it.
Given the information above:
\item[]
\begin{enumerate}[label=\alph*)]
    \item Draw a graph of the “better than” relation for the set G.
    \item Draw a graph of the “better than or equal to” relation for the set G.
    \item Over the set G, for each of the relations (“better than” and “better than or equal to”), is there a least upper bound and a greatest lower bound, and if so what are they?
    \item Over the set G − {A, F }, for each of the relations, what are the minimal and maximal elements?
    \item Are the relations partial orders? Is there any missing information and/or information you are assuming that might affect that answer?
\end{enumerate}
\end{problem}
\begin{solution}{7}
\item[]
\begin{enumerate}[label=\alph*)]
\end{enumerate}
\end{solution}
\pagebreak


\begin{problem}{8}Given $R, S \subseteq A^2$ are relations on $A$. Define $T \subseteq A^2$ such that \\ $xTy \Leftrightarrow (xRy \wedge xSy)$. \\Prove or disprove: If  $R$ and $S$ are equivalence relations,then $T$ is also an equivalence relation.
\end{problem}
\begin{solution}{8}
Suppose $R$ and $S$ are both equivalence relations on a set A. We will show that $R$ and $S$ both being equivalence relations on the set A implies that $T$ is also an equivalence relation.
\vskip 0.2in
It is immediately apparent that since $R$ and $S$ are equivalence relations then:

$x \in A \Rightarrow (xRx)\wedge(xSx)\Rightarrow ((x,x) \in R) \wedge ((x,x) \in S) \Rightarrow (x,x) \in T$
\vskip 0.1in
Thus $x \in A \Rightarrow (x,x) \in T$, therefore $T$ is reflexive.
\vskip 0.2in
Now suppose $(x,y) \in T \Rightarrow (xRy)\wedge(xSy)\Rightarrow(yRx)\wedge(ySx)$ \vskip 0.1in
[Since both $R$ and $S$ are symmetric]
$\Rightarrow((y,x) \in R) \wedge ((y,x) \in S) \Rightarrow (y,x) \in T$
\vskip 0.2in
Since both $R$ and $S$ are symmetric, $T$ will also be symmetric as shown above
\vskip 0.2in
Finally suppose $((x,y) \in T \wedge (y,z) \in T) \Rightarrow ((xRy)\wedge(xSy))\wedge((yRz)\wedge(ySz))\Rightarrow(xRz)\wedge(xSz)$ \vskip 0.1in
[Since both $R$ and $S$ are transitive]
$\Rightarrow((x,z) \in R) \wedge ((x,z) \in S) \Rightarrow (x,z) \in T$
Thus we have shown $T$ is transitive

It follows that $T$ is an equivalence relation since it is reflexive, symmetric and transitive.  $\square$
\end{solution}

\vskip 0.1in

\begin{problem}{9}
 Given poset $(X, \le)$, prove or disprove:
\item[]
\begin{enumerate}[label=\alph*)]
    \item $x \ge y \Leftrightarrow y \le^{\text{--}1} x$
    \item $x \ge y \Leftrightarrow y \le' x$
    \item $x < y \Leftrightarrow y \le' x$
    \item $x > y \Leftrightarrow y (\le^{\text{--}1})' x$
    \item $x > y \Leftrightarrow y (\le')^{\text{--}1} x$
\end{enumerate}
\end{problem}
\begin{solution}{9}
\item[]
\begin{enumerate}[label=\alph*)]
    \item $x \ge y \Leftrightarrow y \le x$ , therefore it cannot imply $y \le^{\text{--}1} x$
    \item $x \ge y \Leftrightarrow y \le x$ , therefore it cannot imply $y \le' x$
    \item $x \ge y \Leftrightarrow y \le x$ , therefore it cannot imply $y \le^{\text{--}1} x$
    \item $x \ge y \Leftrightarrow y \le x$ , therefore it cannot imply $y \le^{\text{--}1} x$
    \item $x \ge y \Leftrightarrow y \le x$ , therefore it cannot imply $y \le^{\text{--}1} x$
\end{enumerate}
\end{solution}

\begin{problem}{10}Given $X \subseteq \N, \forall_{x,y\in\N}\ x \le y \Leftrightarrow \exists_{z \in X} x+z=y$. Prove that,if $(X,\le)$ is a partial order:
\item[]
\begin{enumerate}[label=\alph*)]
    \item $0 \in X$
    \item $\forall_{x,y} (x \in X \wedge y \in X) \rightarrow x + y \in X$
\end{enumerate}
\end{problem}
\begin{solution}{7}
\item[]
\begin{enumerate}[label=\alph*)]
    \item Given $(X,\le)$ is a partial order we can say that $(X,\le)$ is reflexive over $X$,
    thus $$(x \le x \Leftrightarrow \exists_{z \in X} x+z=x) \Rightarrow (z = 0 \wedge z \in X) \Rightarrow 0 \in X$$
    \item Let us assume that the $\forall x,y (x \in X \wedge y \in X) \rightarrow x + y \in X $ is NOT TRUE.\\ Given $(X,\le)$ is a partial order we know $(X,\le)$ is transitive. We also know that $\forall_{x,y\in\N}\ x \le y \Leftrightarrow \exists_{z \in X} x+z=y$ Combining these 2: properties \\ $\forall_{a,b\in X}\ a \le b \Leftrightarrow \exists_{c \in X} a+c=b$ and $\forall_{b,d\in X}\ b \le d \Leftrightarrow \exists_{e \in X} b+e=d \\$ $\Rightarrow \forall_{a,d\in X}\ a \le d \Leftrightarrow \exists_{c+e \in X} a+(c+e)=d$ \\
    This means that for $(X,\le)$ to be transitive $c+e \in X $
\end{enumerate}
\end{solution}

\vskip 0.5in

\begin{problem}{11}Given partially ordered set $(X,\le)$, and $Y \subseteq X$, define a minimum element in $Y$ as: 
\vskip 0.1in
$x$ is a minimum element in $Y \Leftrightarrow x \in Y \wedge \forall_{y \in Y}\ x \le y$
\vskip 0.1in
Note the difference between this definition and that of a minimal element ($x \in Y$ is a minimal element in $Y$ if  $\nexists_{y \in Y}\ y < x$). Prove or disprove:
\item[]
\begin{enumerate}[label=\alph*)]
    \item If x is a minimum element in Y then x is unique.
    \item If x is a minimum element in Y then x is a minimal element in Y .
    \item If x is a minimal element in Y then x is a minimum element in Y .
    \item If every nonempty subset of X has a minimum element, then X is totally ordered.
\end{enumerate}
\end{problem}
\begin{solution}{11}
\item[]
\begin{enumerate}[label=\alph*)]
    \item Let us assume that x is not unique, which means there exists z such that $z \in Y \wedge  \forall_{y \in Y}\ z \le y$ and since $x$ is also a minimum element $x \in Y \wedge \forall_{y \in Y}\ x \le y$ which implies that [after combining those 2 statements]: $x \le z \wedge z \le x \Rightarrow x$ = $z$. Therefore the minimum element is unique
    \item Since $x$ is a minimum element in $Y$ we can say that $ x \in Y \wedge \forall_{y \in Y}\ x \le y$ this clearly implies that there does not exist an element in $Y$ that is smaller than x $\nexists_{y \in Y}\ y < x$.\\ Therefore, x is also the minimal element
    \item Since $x$ is a minimal element in $Y$ we can say that $\nexists_{y \in Y}\ y < x$  this clearly implies that there does not exist an element in $Y$ that is smaller than x $ \Rightarrow \forall_{y \in Y}\ x \le y$.\\ Therefore, x is also the minimum element
    \item
\end{enumerate}
\end{solution}


\vskip 0.5in

\begin{problem}{12}Given finite set A with cardinality N : 
\item[]
\begin{enumerate}[label=\alph*)]
    \item How many distinct functions, $f \colon A \rightarrow A$, exist that map A to A?
    \item How many of these functions are injective? surjective? bijective?
    \item How many distinct binary relations, $R \subseteq A \times A$, exist on set A?
    \item How many of these relations are both symmetric and anti-symmetric?
    \item How many of these relations are both symmetric, anti-symmetric, and reflexive? 
    \item How many of these relations are equivalence relations?
\end{enumerate}
\end{problem}
\begin{solution}{12}
\item[]
\begin{enumerate}[label=\alph*)]
    \item By elementary combinatorics we can say that every element in $A$ has $N$ choices for it's output, therefore total number of function = $N^N$
    \item 
    \begin{enumerate}[label=(\roman*)]
        \item Injective Only: All functions that are not Surjective/ Bijective will be  Injective, therefore the total number of functions are: $N^N - N!$
        \item Surjective Only: Since the size of domain and co-domain is same, all Surjective functions will be bijective by default. Therefore as explained below the total number of functions are: $N!$
        \item Bijective: The first input has N choices, the next one has N-1 choice and so on. Thus the total number of functions are: $N!$
    \end{enumerate}
    \item cardinality($A \times A$) = $N^2$, each element in that set may or may not be in the relation $R$ therefore total number of relations is $2^{N^2}$
    \item Since the relation is both symmetric and anti-symmetric this means $\forall_{x,y \in A}(xRy \Leftrightarrow yRx \wedge x = y)$ therefore the relation is $xRy \Leftrightarrow x = y$. Therefore the number of relations is $2^N$ since we can either choose or not choose any of the ($xRx\ \forall\ x \in A$)
    \item Extending from the previous part, there is only one relation that is symmetric, antisymmetric as well as relexive
    \item 
\end{enumerate}
\end{solution}


% \vskip 0.5in

% \begin{problem}{12}
% \item[]
% \begin{enumerate}[label=\alph*)]
% \end{enumerate}
% \end{problem}
% \begin{solution}{7}
% \item[]
% \begin{enumerate}[label=\alph*)]
% \end{enumerate}
% \end{solution}




\end{document}
